\chapter{Introduction}\label{ch:introduction}
%The introduction is written last, when the work is already finished to a great extent. (If you start with the introduction - a common mistake - it takes much longer and you throw it away later). Its main task is to provide the context for the different classes of readers. You have to win the readers over. The problem the paper deals with should at least be clear in its basic features and appear interesting to the reader. The chapter closes with an overview of the rest of the work. Usually you need at least 4 pages for it, nobody reads more than 10 pages. (4-10pages) (tips for the chapters taken from here: https://tu-dresden.de/ing/informatik/sya/professur-fuer-betriebssysteme/studium/abschlussarbeiten/aufbau-von-diplomarbeiten )

Ultrasound imaging has become an indispensable tool in medical diagnostics due to its non-invasive nature, real-time capabilities, and absence of ionizing radiation. However, the demand for improved image quality, which would make ultrasound more competitive with other imaging modalities, extends beyond the healthcare sector into numerous fields. For example, the need for precise and high-quality imaging data is also evident in mobile communications, radar technology, and industrial automation, all of which rely on accurate data interpretation for critical decision-making. In these fields, as in medicine, achieving high-quality imaging is essential, and image enhancement must balance high accuracy with rapid processing speeds \cite{szabo_diagnostic_2014, michalke_overview_2012}.\par
Improving image quality and processing in ultrasound imaging poses two main challenges. First, hardware improvements to ultrasound transducers are needed to capture more and higher-quality raw data. Second, the increased data volume requires robust, real-time processing capabilities to handle the influx of data without sacrificing speed. As more data is acquired, more computationally intensive reconstruction methods become viable but demand efficient algorithms to maintain real-time processing capabilities. Frequency domain reconstruction, a promising approach in ultrasound imaging, shares computational requirements with techniques used in synthetic aperture radar imaging, where frequency domain processing yields high-resolution images critical for applications like remote sensing and military reconnaissance \cite{dorausch_adoption_2023, alvarez_fourier-based_2014}.\par
One important part for all of this algorithms is the Fourier Transform to enter the frequency domain. Previous research revealed that in ultrasound imaging this is a bottleneck for the realtime capabilities of an imaging algorithm \cite{Richter_2024}. This thesis introduces an optimized algorithm for Fourier transformation, a technique that underpins not only ultrasound image reconstruction but also data processing across various fields. In mobile communications, for instance, frequency analysis via Fourier transforms is critical for demodulating signals and ensuring efficient data transmission. Similarly, Fourier transformations play a crucial role in radar imaging, where they facilitate the analysis of frequency-shifted signals to produce detailed, real-time imagery. By enhancing Fourier transformation performance, the proposed algorithm aims to benefit a wide range of applications beyond medical imaging. \cite{alvarez_fourier-based_2014, gupta_fourier_2013}\par
The algorithm presented in this thesis is designed with mobile platform constraints in mind, specifically targeting a \ac{soc} configuration rather than larger, more powerintensive GPUs. Such a mobile platform would reduce hardware form factors, opening up possibilities for ultrasound technology in non-clinical settings such as ambulances and mobile medical units. This aligns with broader trends in technology miniaturization seen in wearable devices and mobile industrial sensors, where \ac{soc} solutions enable advanced computation in a compact form \cite{michalke_overview_2012, kim_miniaturization_2022}.\par
The proposed work covers the design and prototype implementation of this new algorithm, with benchmarks assessing its real-time capabilities. Using the Xilinx Versal VCK190 as the hardware platform, this thesis leverages the AI acceleration capabilities of the \ac{soc}, exploring recent advancements in chip design to push the boundaries of real-time Fourier transformation performance on mobile platforms. This exploration could have implications not only for ultrasound imaging but also for other data-intensive applications in fields requiring efficient frequency-domain processing.\par
This thesis is structured as follows: Chapter 2 provides background on the imaging algorithm and the mathematical foundations, as well as an overview of the VCK190 hardware. Chapter 3 reviews related work in the field, including independent research projects and implementations by Xilinx as part of the VCK190 development and support libraries. Chapter 4 details the design of the proposed algorithm, discussing various design choices and comparing them to alternative approaches. Chapter 5 covers the implementation process, including the generation of synthetic test data for benchmarking. Chapter 6 presents the benchmark results and highlights some unexpected findings. Chapter 7 outlines the limitations of this work and suggests directions for future research to enhance the algorithm. Finally, Chapter 8 provides a summary of the thesis accomplishments.\par